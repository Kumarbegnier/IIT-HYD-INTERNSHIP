\documentclass[journal,12pt,twocolumn]{IEEEtran}

\usepackage{setspace}
\usepackage{gensymb}

\singlespacing


\usepackage[cmex10]{amsmath}

\usepackage{amsthm}

\usepackage{mathrsfs}
\usepackage{txfonts}
\usepackage{stfloats}
\usepackage{bm}
\usepackage{cite}
\usepackage{cases}
\usepackage{subfig}

\usepackage{longtable}
\usepackage{multirow}

\usepackage{enumitem}
\usepackage{mathtools}
\usepackage{steinmetz}
\usepackage{tikz}
\usepackage{circuitikz}
\usepackage{verbatim}
\usepackage{tfrupee}
\usepackage[breaklinks=true]{hyperref}
\usepackage{graphicx}
\usepackage{tkz-euclide}

\usetikzlibrary{calc,math}
\usepackage{listings}
\usepackage{color} %%
\usepackage{array} %%
\usepackage{longtable} %%
\usepackage{calc} %%
\usepackage{multirow} %%
\usepackage{hhline} %%
\usepackage{ifthen} %%
\usepackage{lscape}
\usepackage{multicol}
\usepackage{chngcntr}

\DeclareMathOperator*{\Res}{Res}

\renewcommand\thesection{\arabic{section}}
\renewcommand\thesubsection{\thesection.\arabic{subsection}}
\renewcommand\thesubsubsection{\thesubsection.\arabic{subsubsection}}

\renewcommand\thesectiondis{\arabic{section}}
\renewcommand\thesubsectiondis{\thesectiondis.\arabic{subsection}}
\renewcommand\thesubsubsectiondis{\thesubsectiondis.\arabic{subsubsection}}



\def\inputGnumericTable{} %%

\lstset{
%language=C,
frame=single,
breaklines=true,
columns=fullflexible
}
\begin{document}


\newtheorem{theorem}{Theorem}[section]
\newtheorem{problem}{Problem}
\newtheorem{proposition}{Proposition}[section]
\newtheorem{lemma}{Lemma}[section]
\newtheorem{corollary}[theorem]{Corollary}
\newtheorem{example}{Example}[section]
\newtheorem{definition}[problem]{Definition}

\newcommand{\BEQA}{\begin{eqnarray}}
\newcommand{\EEQA}{\end{eqnarray}}
\newcommand{\define}{\stackrel{\triangle}{=}}
\bibliographystyle{IEEEtran}
\providecommand{\mbf}{\mathbf}
\providecommand{\pr}[1]{\ensuremath{\Pr\left(#1\right)}}
\providecommand{\qfunc}[1]{\ensuremath{Q\left(#1\right)}}
\providecommand{\sbrak}[1]{\ensuremath{{}\left[#1\right]}}
\providecommand{\lsbrak}[1]{\ensuremath{{}\left[#1\right.}}
\providecommand{\rsbrak}[1]{\ensuremath{{}\left.#1\right]}}
\providecommand{\brak}[1]{\ensuremath{\left(#1\right)}}
\providecommand{\lbrak}[1]{\ensuremath{\left(#1\right.}}
\providecommand{\rbrak}[1]{\ensuremath{\left.#1\right)}}
\providecommand{\cbrak}[1]{\ensuremath{\left\{#1\right\}}}
\providecommand{\lcbrak}[1]{\ensuremath{\left\{#1\right.}}
\providecommand{\rcbrak}[1]{\ensuremath{\left.#1\right\}}}
\theoremstyle{remark}
\newtheorem{rem}{Remark}
\newcommand{\sgn}{\mathop{\mathrm{sgn}}}
\providecommand{\abs}[1]{\l\vert#1\r\vert}
\providecommand{\res}[1]{\Res\displaylimits_{#1}}
\providecommand{\norm}[1]{\lVert#1\rVert}
%\providecommand{\norm}[1]{\lVert#1\rVert}
\providecommand{\mtx}[1]{\mathbf{#1}}
\providecommand{\mean}[1]{E\l[ #1 \r]}
\providecommand{\fourier}{\overset{\mathcal{F}}{ \rightleftharpoons}}
%\providecommand{\hilbert}{\overset{\mathcal{H}}{ \rightleftharpoons}}
\providecommand{\system}{\overset{\mathcal{H}}{ \longleftrightarrow}}
%\newcommand{\solution}[2]{\textbf{Solution:}{#1}}
\newcommand{\solution}{\noindent \textbf{Solution: }}
\newcommand{\cosec}{\,\text{cosec}\,}
\providecommand{\dec}[2]{\ensuremath{\overset{#1}{\underset{#2}{\gtrless}}}}
\newcommand{\myvec}[1]{\ensuremath{\begin{pmatrix}#1\end{pmatrix}}}
\newcommand{\mydet}[1]{\ensuremath{\begin{vmatrix}#1\end{vmatrix}}}
\numberwithin{equation}{subsection}
\makeatletter
\@addtoreset{figure}{problem}
\makeatother
\let\StandardTheFigure\thefigure
\let\vec\mathbf
\renewcommand{\thefigure}{\theproblem}
\def\putbox#1#2#3{\makebox[0in][l]{\makebox[#1][l]{}\raisebox{\baselineskip}[0in][0in]{\raisebox{#2}[0in][0in]{#3}}}}
\def\rightbox#1{\makebox[0in][r]{#1}}
\def\centbox#1{\makebox[0in]{#1}}
\def\topbox#1{\raisebox{-\baselineskip}[0in][0in]{#1}}
\def\midbox#1{\raisebox{-0.5\baselineskip}[0in][0in]{#1}}
\vspace{3cm}
\title{Assignment 2}
\author{Neeraj Kumar}
\maketitle
\newpage
\bigskip
\renewcommand{\thefigure}{\theenumi}
\renewcommand{\thetable}{\theenumi}
Download all python codes from
\begin{lstlisting}
https://github.com/
\end{lstlisting}

%
and latex-tikz codes from
%
\begin{lstlisting}
https://github.com/
\end{lstlisting}
\section{Question No. 2.3 - Quadratic forms}
Find the locus of all the unit vectors in the
xy-plane%
\section{Solution}
% 
let,the unit vector be $\vec{a}=1$, $OA= \vec{a}$ \\
%
We know that,
%
\begin{equation}
\vec{a} = x\vec{i}+y\vec{j}+z\vec{k}
\end{equation}
%
since the vector in xy plane, there is no z-coordinate.Hence,
%
\begin{equation}
\vec{a} = x\vec{i}+y\vec{j}
\end{equation}
%
Taking a general vector\\
Angle AOB with x-axis in between is  $\vec{a}$ and $\vec{i}$ is $\angle{AOB}$
%
\begin{figure}[!h]
\centering
\includegraphics[width= \columnwidth]{figure2.png}
\caption{unit vector}
\label{Figure}
\end{figure}
%
we know that,
%
\begin{equation}
\vec{a}.\vec{b} = |\vec{a}||\vec{b}|\cos{AOB}
\end{equation}
%
\begin{equation}
\vec{a}.\vec{b} = |\vec{a}||\vec{b}|\cos{AOB}
\end{equation}
%
putting $a=a$, $b=i$ , as we know that a is unit vector
\begin{equation}
    \vec{a}.\vec{i} = \cos{AOB}
\end{equation}
%
\begin{equation}
 (x\vec{i}+y\vec{j}+z\vec{k}).\vec{i} = \cos{AOB} 
\end{equation}
%
\begin{equation}
 (x\vec{i}+y\vec{j}+z\vec{0})(x\vec{1}+y\vec{0}+z\vec{0}) = \cos{AOB} 
\end{equation}
%
\begin{equation}
x= \cos{AOB}
\end{equation}
%
Angle with y-axis, in between $\vec{a}$,  and  $\vec{j}$  is $(90\degree - \angle{AOB})$\\
%
so, angle between \\
%
\begin{equation}
\vec{a}.\vec{j} = |\vec{a}| |\vec{j}| (90\degree -\cos{AOB})
\end{equation}
%
\begin{equation}
\vec{a}.\vec{j} = (\sin{AOB})
\end{equation}
%
\begin{equation}
 (x\vec{i}+y\vec{j}+z\vec{k}).\vec{j} = \cos{AOB} 
\end{equation}
%
\begin{equation}
 (x\vec{i}+y\vec{j}+z\vec{0})(x\vec{0}+y\vec{1}+z\vec{0}) = \cos{AOB} 
\end{equation}
%
\begin{equation}
y= \sin{AOB}
\end{equation}
%
Thus,
%
\begin{equation}
\vec{a} = x\vec{i}+y\vec{j}
\end{equation}
%
\begin{equation}
\vec{a} = (\cos{AOB})\vec{i}+(\cos{AOB}) \vec{j}
\end{equation}
%
This value will be in all quadrants($0\degree$ to $360\degree$)
\\


\end{document}