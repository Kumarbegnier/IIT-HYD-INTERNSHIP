\documentclass[journal,12pt,twocolumn]{IEEEtran}
%
\usepackage{setspace}
\usepackage{gensymb}
\singlespacing
\usepackage[cmex10]{amsmath}
\usepackage{siunitx}
\usepackage{amsthm}

\usepackage{mathrsfs}

\usepackage{txfonts}
\usepackage{stfloats}

\usepackage{steinmetz}
\usepackage{cite}
\usepackage{cases}
\usepackage{subfig}
\usepackage{longtable}
\usepackage{multirow}
\usepackage{enumitem}
\usepackage{mathtools}
\usepackage{tikz}
\usepackage{circuitikz}
\usepackage{verbatim}
\usepackage{tfrupee}
\usepackage[breaklinks=true]{hyperref}
\usepackage{tkz-euclide} % loads  TikZ and tkz-base
\usetikzlibrary{calc,math}
\usetikzlibrary{fadings}
\usepackage{listings}
    \usepackage{color}                                            %%
    \usepackage{array}                                            %%
    \usepackage{longtable}                                        %%
    \usepackage{calc}                                             %%
    \usepackage{multirow}                                         %%
    \usepackage{hhline}                                           %%
    \usepackage{ifthen}                                           %%
  %optionally (for landscape tables embedded in another document): %%
    \usepackage{lscape}     
\usepackage{multicol}
\usepackage{chngcntr}
\DeclareMathOperator*{\Res}{Res}

\renewcommand\thesection{\arabic{section}}
\renewcommand\thesubsection{\thesection.\arabic{subsection}}
\renewcommand\thesubsubsection{\thesubsection.\arabic{subsection}}

\renewcommand\thesectiondis{\arabic{section}}
\renewcommand\thesubsectiondis{\thesectiondis.\arabic{subsubsection}}
\renewcommand\thesubsubsectiondis{\thesubsectiondis.\arabic{subsubsection}}

\hyphenation{op-tical net-works semi-conduc-tor}
\def\inputGnumericTable{}                                 %%

\lstset{
%language=C,
frame=single, 
breaklines=true,
columns=fullflexible
}
\begin{document}
%


\newtheorem{theorem}{Theorem}[section]
\newtheorem{problem}{Problem}
\newtheorem{proposition}{Proposition}[section]
\newtheorem{lemma}{Lemma}[section]
\newtheorem{corollary}[theorem]{Corollary}
\newtheorem{example}{Example}[section]
\newtheorem{definition}[problem]{Definition}

\newcommand{\BEQA}{\begin{eqnarray}}
\newcommand{\EEQA}{\end{eqnarray}}
\newcommand{\define}{\stackrel{\triangle}{=}}
\bibliographystyle{IEEEtran}
\providecommand{\mbf}{\mathbf}
\providecommand{\pr}[1]{\ensuremath{\Pr\left(#1\right)}}
\providecommand{\qfunc}[1]{\ensuremath{Q\left(#1\right)}}
\providecommand{\sbrak}[1]{\ensuremath{{}\left[#1\right]}}
\providecommand{\lsbrak}[1]{\ensuremath{{}\left[#1\right.}}
\providecommand{\rsbrak}[1]{\ensuremath{{}\left.#1\right]}}
\providecommand{\brak}[1]{\ensuremath{\left(#1\right)}}
\providecommand{\lbrak}[1]{\ensuremath{\left(#1\right.}}
\providecommand{\rbrak}[1]{\ensuremath{\left.#1\right)}}
\providecommand{\cbrak}[1]{\ensuremath{\left\{#1\right\}}}
\providecommand{\lcbrak}[1]{\ensuremath{\left\{#1\right.}}
\providecommand{\rcbrak}[1]{\ensuremath{\left.#1\right\}}}
\theoremstyle{remark}
\newtheorem{rem}{Remark}
\newcommand{\sgn}{\mathop{\mathrm{sgn}}}
\providecommand{\abs}[1]{\$left\vert#1\$right\vert}
\providecommand{\res}[1]{\Res\displaylimits_{#1}} 
\providecommand{\norm}[1]{\$left\lVert#1\$right\rVert}
%\providecommand{\norm}[1]{\lVert#1\rVert}
\providecommand{\mtx}[1]{\mathbf{#1}}
\providecommand{\mean}[1]{E\$left[ #1 \$right]}
\providecommand{\fourier}{\overset{\mathcal{F}}{ \rightleftharpoons}}
%\providecommand{\hilbert}{\overset{\mathcal{H}}{ \rightleftharpoons}}
\providecommand{\system}{\overset{\mathcal{H}}{ \longleftrightarrow}}
	%\newcommand{\solution}[2]{\textbf{Solution:}{#1}}
\newcommand{\solution}{\noindent \textbf{Solution: }}
\newcommand{\cosec}{\,\text{cosec}\,}
\providecommand{\dec}[2]{\ensuremath{\overset{#1}{\underset{#2}{\gtrless}}}}
\newcommand{\myvec}[1]{\ensuremath{\begin{pmatrix}#1\end{pmatrix}}}
\newcommand{\mydet}[1]{\ensuremath{\begin{vmatrix}#1\end{vmatrix}}}
\numberwithin{equation}{subsection}
\makeatletter
\@addtoreset{figure}{problem}
\makeatother
\let\StandardTheFigure\thefigure
\let\vec\mathbf
\renewcommand{\thefigure}{\theproblem}
\def\putbox#1#2#3{\makebox[0in][l]{\makebox[#1][l]{}\raisebox{\baselineskip}[0in][0in]{\raisebox{#2}[0in][0in]{#3}}}}
     \def\rightbox#1{\makebox[0in][r]{#1}}
     \def\centbox#1{\makebox[0in]{#1}}
     \def\topbox#1{\raisebox{-\baselineskip}[0in][0in]{#1}}
     \def\midbox#1{\raisebox{-0.5\baselineskip}[0in][0in]{#1}}
\vspace{3cm}
\title{ASSIGNMENT-1}
\author{Neeraj Kumar}
\maketitle
\newpage
\bigskip
\renewcommand{\thefigure}{\theenumi}
\renewcommand{\thetable}{\theenumi}
%
download all python codes from 
%
\begin{lstlisting}
https://github.com/Kumarbegnier/IIT-HYD-INTERNSHIP/tree/main/ASSIGNMENT_201/code
\end{lstlisting}
%
latex-tikz codes from
%
\begin{lstlisting}
https://github.com/Kumarbegnier/IIT-HYD-INTERNSHIP/blob/main/ASSIGNMENT_201/Latex.tex
\end{lstlisting}
%
\section{QUESTION NO-2.19}
 \noindent If  $XY$ = 6,$\angle{X}$ = \ang{30} and $\angle{Y}$ = \ang{100}.Can you draw a triangle?
%
\section{Solution}
%
\begin{equation}
               Given, XY=6 \hspace{0.25cm} \angle X=30\si{\degree}  \hspace{0.25cm} \angle Y=100\si{\degree}
\end{equation}
%
\begin{align}
Let, \hspace{0.25cm} XY=z, \hspace{0.25cm} YZ=x, \hspace{0.25cm} XZ=y
\end{align}
%
\begin{figure}[!b]
 \centering
           \includegraphics[width=\columnwidth]{Figure_1.png}
           \caption{Constructed Triangle}
           \label{Figure}
\end{figure}
%
Angle  Sum  Property
%
\begin{align}
             \angle Z\si{\degree} = \angle 180\si{\degree} - \angle X\si{\degree} + \angle Y\si{\degree}\\
             \angle Z\si{\degree} = \angle 50\si{\degree}
\end{align}
%
To find the side $y$ by using the formula
%
\begin{align}
\frac{\sin{X}}{x} = \frac{\sin{Y}}{y} = \frac{\sin{Z}}{z}
\end{align}
%
written as,
%
\begin{align}
y&= z\brak{\frac{\sin{Y}}{\sin{Z}}} 
&= 6 \brak{\frac{\sin{\ang{100}}}{\sin{\ang{50}}}}
&= 7.7134
\end{align}
In the $\triangle XYZ$, vertex of Y can be expressed in polar coordinate.
\begin{equation}
\vec{X} = \myvec{\ 0\\ 0}, \hspace{0.25cm} \vec{Y} = \myvec{ \cos X\si{\degree}\\ \sin X\si{\degree}} ,\hspace{0.25cm} \vec{Z} =\myvec{y \\ 0} \\
\\
\end{equation}
\begin{align}
\vec{X} =\myvec{0 \\ 0},
\hspace{0.25cm} \vec{Y} =6\myvec{\cos{\ang{30}} \\ \sin{\ang{30}}} = 6\myvec{\sqrt{3}/2 \\ 0.5 }, 
\hspace{0.25cm} \vec{Z} =\myvec{0.50 \\ 0}
\end{align}
The values of X, Y and Z are substituted and the triangle is plotted as given above.
\end{document}
